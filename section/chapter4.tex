\chapter{Pertemuan 4}

\section{Issues \#31}
Pada \textit{issues \#31} (Menambahkan \textit{Casting} pada InputData2.java) casting ini dilakukan untuk mensingkronisasi view pada file activity\_InputData2.XML dengan InputData2.java.
\begin{verbatim}
jenispengeluaran = findViewById(R.id.jenispengeluaran);
\end{verbatim}
jenis pengeluaran disini merupakan \textit{variable} yang dapat dipanggil. Dalam \textit{variable} tersebut terdapat \textit{findViewById} yang digunakan untuk mencari view pada activity\_InputData2.XML dan \textit{R.id.jenispengeluaran} merupakan id dari view yang akan di sinkronisasi pada activity\_InputData2.XML

\section{Issues \#32} 
Pada \textit{issues \#32} (penambahan perintah SQL) Perintah SQL ini ditambahkan untuk melakukan input data ke database yang diinputkan pada DatabaseHelper.java
\begin{verbatim}
    db.execSQL("create table "+ TBLNAME +"(ID INTEGER PRIMARY KEY AUTOINCREMENT, JENIS_PEMASUKAN STRING," +
                "PEMASUKAN NUMBER, JENIS_PENGELUARAN STRING, PENGELUARAN NUMBER, TANGGAL STRING)");
                "PEMASUKAN NUMBER, JENIS_PENGELUARAN STRING, PENGELUARAN NUMBER, " +
                "TANGGAL_PEMASUKAN STRING, TANGGAL_PENGELUARAN STRING)"); 
\end{verbatim}

\section{Issues \#33}
Pada \textit{issues \#33} (\textit{Error: setVibrate(long[]) in Builder cannot be applied to ()}) disebabkan karena perintah .setVibrate(); hanya bisa menerima tipe data Array. Maka melakukan perubahan kode seperti ini:
\begin{verbatim}
    long[] PolaGetar = {100, 100};
    .setVibrate(PolaGetar);
\end{verbatim}

\section{Issues \#34}
Pada \textit{issues \#34}(mengubah gambar pada notifikasi). Pengubahan atau penambahan gambar pada notifikasi dengan menambahkan baris kode seperti dibawah pada method notifikasi.
\begin{verbatim}
    .setSmallIcon(R.drawable.ic_cash)
\end{verbatim}

\section{Issues \#35}
Pada \textit{issues \#35} (\textit{Cannot resolve method 'setContentTitle(java.lang.string)'}) Penyebabnya adalah Kekurangan tanda kurung tutup pada
\begin{verbatim}
    .setSmallIcon(R.drawable.ic_cash
    .setContentTitle("Issues Notif")
\end{verbatim}
solusinya dengan menambahkan kurung tutup pada 
\begin{verbatim}
    .setSmallIcon(R.drawable.ic_cash)
\end{verbatim}

\section{Issues \#36}
Pada \textit{issues \#36} (\textit{Error: ) expected}). Penyebabnya adalah kurangnya tanda ) pada 
\begin{verbatim}
    notificationManager.notify(NOTIFICATION_ID,builder.build();
\end{verbatim}
Solusinya menambahkan kurung tutup pada baris kode tersebut seperti:
\begin{verbatim}
    notificationManager.notify(NOTIFICATION_ID,builder.build());
\end{verbatim}

\section{Issues \#37}
Pada \textit{issues \#37}(\textit{Array initializer is not allowed here}).
\begin{verbatim}
    long PolaGetar = {100, 100};
\end{verbatim}
Karena pada baris tersebut tidak ada tanda [] setelah long yang menandakan bahwa tipe data tersebuat adalah array.\\
Solusinya adalah dengan menambahkan [] setelah long untuk menandakan bahwa data tersebut merupakan array.
\begin{verbatim}
    long [] PolaGetar = {100, 100};
\end{verbatim}

\section{Issues \#38}
Pada \textit{issues \#38} (\textit{Error: when INSERT INTO tabungan}). Error ini terjadi ketika akan memasukan data input pemasukan ke dalam tabel tabungan pada field Jenis\_pemasukan, Jumlah\_pemasukan, tanggal. Hal ini terjadi karena field tanggal tidak ada.
Solusi, menambahkan field tanggal\_pemasukan pada DatabaseHelper
\begin{verbatim}
    public static final String COL6 = "TANGGAL_PEMASUKAN";
\end{verbatim}

\section{Issues \#39}
Pada \textit{issues \#39}(Mengubah Format Penulisan Tanggal). Untuk mengubah format penulisan tanggal menggunakan perintah yang dimasukan pada input data.java
\begin{verbatim}
    final Calendar calendar = Calendar.getInstance();
    String currentDate = DateFormat.getDateInstance(DateFormat.FULL)
    .format(calendar.getTime());

    myDB = new DatabaseHelper(this);


    calendar.set(Calendar.YEAR, year);
                calendar.set(Calendar.MONTH, month);
                calendar.set(Calendar.DAY_OF_MONTH, dayOfMonth);
                String date = DateFormat.getDateInstance(DateFormat.FULL)
                .format(calendar.getTime());
\end{verbatim}

\section{Issues \#40}
Pada \textit{issues \#40}(Menambahkan perintah SQL pembuatan tabel). Perintah SQL ini ditulis dalam method public void onCreate(SQLiteDatabase db) pada DatabaseHelper.java
\begin{verbatim}
    public void onCreate(SQLiteDatabase db) {
        db.execSQL("create table "+ TBLNAME +
        "(ID_PEMASUKAN INTEGER PRIMARY KEY AUTOINCREMENT," + 
        "JENIS_PEMASUKAN STRING, PEMASUKAN NUMBER," +
         "TANGGAL_PEMASUKAN STRING)");

        db.execSQL("create table "+ TBLNAME2 +
        "(ID_PENGELUARAN INTEGER PRIMARY KEY AUTOINCREMENT," +
        "JENIS_PENGELUARAN STRING, PENGELUARAN NUMBER," + 
        "TANGGAL_PENGELUARAN STRING)");
    }
\end{verbatim}