\chapter{Pertemuan 5}

\section{Issues \#41}
Pada \textit{issues \#41}(Menambahkan perintah pada onUpgrade database helper). Penambahan perintah yang dilakukan ini agar database di-replace ketika sudah ada filenya pada perangkat yang digunakan. PErintah ini ditulis pada DatabaseHelper
\begin{verbatim}
    @Override
    public void onUpgrade(SQLiteDatabase db, int i, int i1) {
        db.execSQL("DROP TABLE IF EXISTS "+TBLNAME);
        db.execSQL("DROP TABLE IF EXISTS "+TBLNAME2);
        onCreate(db);
    }
\end{verbatim}

\section{Issues \#42}
Pada \textit{issues \#42}(Mengubah list.java). Pengubahan pada list ini digunakan untuk meisahkan antara list pemasukan dan pengeluaran 
\begin{verbatim}
    if (cursor.getCount() == 0) {
        Toast.makeText(this, "TIDAK ADA DATA", Toast.LENGTH_SHORT)
        .show();
    } else {
        while (cursor.moveToNext()) {
            listitem.add("\n" + "Tanggal : " + cursor.getString(3) 
                        + "\n" + "\n" + "Jenis Pemasukan : " 
                        + cursor.getString(1)
                        + "\n" + "Jumlah : Rp " + cursor.getString(2) 
                        + ",00" + "\n");
            }
            final ArrayAdapter adapter = new ArrayAdapter<>
            (this, android.R.layout.simple_list_item_1, listitem);
            listView.setAdapter(adapter);
\end{verbatim}

\section{Issues \#43}
Pada \textit{issues \#43}(Menambahkan List2.java) Penambahan ini dilakukan melalui \textit{new} pada aplikasi android studio.
New -- activity -- empty activity

\section{Issues \#44}
Pada \textit{issues \#44}(Menambahkan ListView pada activity\_list2.xml). Untuk menambahkan listView ini, dengan cara \textit{drag n drop} melalui palletes yang ada pada desain activity\_list2.xml
Namun Bisa juga dengan melakukannya secara manual dengan menginputkan baris kode ini:
\begin{verbatim}
    <com.baoyz.swipemenulistview.SwipeMenuListView
    android:id="@+id/listView2"
    android:layout_width="match_parent"
    android:layout_height="match_parent"/>
\end{verbatim}

\section{Issues \#45}
Pada \textit{issues \#45}(Menambahkan perintah untuk menampilkan data pada List2.java). Penambahan perintah untuk menampilkan data pada arraylist yang diambil dari database.

\section{Issues \#46}
Pada \textit{issues \#46}(\textit{Error: Cannot Resolve Method tampilkanDataKeluar()}). Error ini terjadi karena method yang dipanggil tidak ada, maka kita harus membuat methodnya pada DatabaseHelper.java
\begin{verbatim}
    public Cursor tampilkanDataKeluar() {
        SQLiteDatabase db = this.getReadableDatabase();
        Cursor res;
        res = db.rawQuery("SELECT * FROM " + TBLNAME2, null);
        return res;
    }
\end{verbatim}

\section{Issues \#47}
Pada \textit{issues \#47}(\textit{Error: ; expected}). Error ini terjadi karena kekurangan tanda ; pada akhir baris kode.
\begin{verbatim}
    public Cursor tampilkanDataKeluar() { 
    SQLiteDatabase db = this.getReadableDatabase();
    Cursor res;
    res = db.rawQuery("SELECT * FROM "+TBLNAME2,null);
    return 
\end{verbatim}
Solusi: Menambahkan tanda ; pada akhir baris kode setelah perintah \textit{retutn}
\begin{verbatim}
    return;
\end{verbatim}

\section{Issues \#48}
Pada \textit{issues \#48}(\textit{Error: '\}' expected}). Error ini terjadi karena kekurangan tanda kurung tutup (\}) pada akhir method atau prosedur.
\begin{verbatim}
    public Cursor tampilkanDataKeluar() {
        SQLiteDatabase db = this.getReadableDatabase();
        Cursor res;
        res = db.rawQuery("SELECT * FROM " + TBLNAME2, null);
        return res;
\end{verbatim}
Solusi: Menambahkan \} pada akhir method
\begin{verbatim}
    public Cursor tampilkanDataKeluar() {
        SQLiteDatabase db = this.getReadableDatabase();
        Cursor res;
        res = db.rawQuery("SELECT * FROM " + TBLNAME2, null);
        return res;
    }
\end{verbatim}

\section{Issues \#49}
Pada \textit{issues \#49}(Set Tanggal Otomatis Hari Ini ).
Untuk set tanggal otomatis hari ini dengan menuliskan 
\begin{verbatim}
    tglpengeluaran.setText(currentDate);
\end{verbatim}
Pada inputData2.java

\section{Issues \#50}
Pada \textit{issues \#50}